% !TEX TS-program = XeLaTeX
% !TEX spellcheck = en-US
\documentclass[aspectratio=169]{beamer}

% --- Themes and Colors ---
\usetheme{example} % Changed to a standard theme (replace with 'example' if you have a custom file)
\usecolortheme{default}
\setbeamertemplate{navigation symbols}{} % Remove navigation buttons at bottom

% --- Packages ---
\usepackage{booktabs} % For professional looking tables
\usepackage{graphicx}
\usepackage{hyperref}
\usepackage{xcolor}

% --- Global Formatting Tweaks ---
% Add breathing room between bullet points
\setbeamertemplate{itemize/enumerate body begin}{\setlength{\itemsep}{0.8em}}
\setbeamertemplate{itemize/enumerate subbody begin}{\setlength{\itemsep}{0.5em}}

% --- Title Information ---
\title{Lecture 1: \\ Introduction and Working with Data}
\institute{GRA 4160: Predictive Modelling with Machine Learning}
\date{January 7th 2027}
\author{Vegard H\o ghaug Larsen}

\begin{document}

% -----------------------------------------------------------------------------
% Title Slide
% -----------------------------------------------------------------------------
\begin{frame}
    \titlepage
\end{frame}

% -----------------------------------------------------------------------------
% About Me
% -----------------------------------------------------------------------------
\begin{frame}{About me}
    \begin{columns}
        \begin{column}{0.9\textwidth}
            \textbf{Vegard H\o ghaug Larsen} \\
            \small (Department of Data Science and Analytics)
            
            \vspace{1em}
            
            \textbf{Background:} \\
            PhD in Economics followed by six years in Norges Bank.
            
            \vspace{1em}
            
            \textbf{Contact:} \\
            Email: \href{mailto:vegard.h.larsen@bi.no}{vegard.h.larsen@bi.no} \\
            Office: B3Y-075

            \vspace{1em}

            \textbf{Research} \\
            \small My work is at the intersection of economics and data science, where I use machine learning and natural language processing techniques to study the transmission of economic shocks, understand how agents form their expectations, and develop methods to measure unobserved concepts such as sentiment, uncertainty, and climate risk.
        \end{column}
    \end{columns}
\end{frame}

% -----------------------------------------------------------------------------
% Syllabus
% -----------------------------------------------------------------------------
\begin{frame}{Syllabus}
    \begin{columns}[T] % T aligns columns at the top
        \begin{column}{0.65\textwidth}
            \textbf{Primary Textbook:} \\
            \textit{The Elements of Statistical Learning} \\
            \small by Hastie, Tibshirani, \& Friedman \\
            (\href{https://hastie.su.domains/Papers/ESLII.pdf}{\color{blue}\underline{Free PDF Link}})
            
            \vspace{1.5cm} % Vertical gap between books
            
            \textbf{Gentle Introduction:} \\
            \textit{An Introduction to Statistical Learning} \\
            \small by James, Witten, Hastie, \& Tibshirani \\
            (\href{https://www.statlearning.com/}{\color{blue}\underline{Free PDF Link}})
        \end{column}
        
        \begin{column}{0.3\textwidth}
            % Ensure these image paths are correct on your machine
            % Using 'draft' option prevents crash if images are missing
            \includegraphics[width=0.5\textwidth, keepaspectratio]{figures/ESL.jpg}
            \vspace{0.5cm}
            \includegraphics[width=0.5\textwidth, keepaspectratio]{figures/ISL.jpg}
        \end{column}
    \end{columns}
\end{frame}

% -----------------------------------------------------------------------------
% Prerequisites
% -----------------------------------------------------------------------------
\begin{frame}{Prerequisites / What I Expect You to Know}
    \begin{block}{Foundations}
        \begin{itemize}
            \item \textbf{Mathematics:} Basic linear algebra and multivariate calculus.
            \item \textbf{Statistics:} Fundamental probability and statistical concepts.
        \end{itemize}
    \end{block}
    
    \pause
    
    \begin{block}{Technical Skills}
        \begin{itemize}
            \item \textbf{Python Programming:} Experience with NumPy, Pandas, and Matplotlib.
            \item \textbf{OOP:} Ability to write and understand Object-Oriented structures.
        \end{itemize}
    \end{block}
\end{frame}

% -----------------------------------------------------------------------------
% About this Course
% -----------------------------------------------------------------------------
\begin{frame}{About this Course}
    \begin{itemize}
        \item Focused on \textbf{predictive modeling} using traditional (pre-deep learning) methods.
        \pause
        \item \textbf{Core Algorithms:} Linear/logistic regression, decision trees, and ensemble methods.
        \pause
        \item \textbf{Hands-on Practice:} Writing code from scratch to understand mechanics.
        \pause
        \item \textbf{Real Workflows:} Preprocessing, training, tuning, and evaluation.
        \pause
        \item \textbf{Goal:} Critical evaluation and improvement of models for real-world tasks.
    \end{itemize}
\end{frame}

% -----------------------------------------------------------------------------
% Mini Project
% -----------------------------------------------------------------------------
\begin{frame}{Mini Project}
    \begin{columns}[t]
        \begin{column}{0.55\textwidth}
            \textbf{Logistics}
            \begin{itemize}
                \item \textbf{Group Work:} 2--4 students (Form groups by next week).
                \item \textbf{Objective:} Apply methods to a real-world problem.
                \item \textbf{Feedback:} Up to 2 scheduled sessions per group.
                \item \textbf{GitHub:} Recommended for sharing code/report.
            \end{itemize}
        \end{column}
        
        \begin{column}{0.42\textwidth}
            \begin{alertblock}{Evaluation}
                \begin{itemize}
                    \item \textbf{Weight:} 30\% of final grade.
                    \item \textbf{Presentation:} End of semester (External grader).
                    \item \textbf{Exam:} Project topics are relevant for the final exam.
                \end{itemize}
            \end{alertblock}
        \end{column}
    \end{columns}
\end{frame}

% -----------------------------------------------------------------------------
% Course Structure
% -----------------------------------------------------------------------------
\begin{frame}{Course Structure}
    \begin{itemize}
        \item \textbf{Interactive Format:} Minimal lecture-style; focus on live coding.
        \pause
        \item \textbf{Typical Session:} 
        \begin{enumerate}
            \item Brief concept intro (slides)
            \item Live coding \& examples
            \item Exercises
        \end{enumerate}
        \pause
        \item \textbf{Materials:} Background notebooks + Exercise notebooks.
        \pause
        \item \textbf{Expectation:} Review notebooks \textit{before} class.
    \end{itemize}
\end{frame}

% -----------------------------------------------------------------------------
% Generative AI
% -----------------------------------------------------------------------------
\begin{frame}{Using Generative AI as a Learning Tool}
    \begin{itemize}
        \item \textbf{Role:} Use GenAI as a learning partner, not a replacement for reasoning.
        \pause
        \item \textbf{Usage:} Ask for help with implementing code, clarifications, alternative solutions, and code explanations.
        \pause
        \item \textbf{Tools:} 
        \begin{itemize}
            \item \textbf{Google Gemini:} Access to BI-students.
            \item \textbf{GitHub Copilot:} Free for students, integrated with VSCode.
        \end{itemize}
        \pause
        \item \textbf{Warning:} Always apply critical thinking. You are responsible for the output.
    \end{itemize}
\end{frame}

% -----------------------------------------------------------------------------
% IDE / Tools
% -----------------------------------------------------------------------------
\begin{frame}{Tools: Integrated Development Environment (IDE)}
    \begin{columns}
        \begin{column}{0.5\textwidth}
            \textbf{Why use an IDE?}
            \begin{itemize}
                \item Integrated tools (linting, debugging).
                \item Higher productivity.
            \end{itemize}
        \end{column}
        
        \begin{column}{0.5\textwidth}
            \textbf{Recommendation: VS Code}
            \begin{itemize}
                \item Lightweight \& Cross-platform.
                \item Excellent Python \& Jupyter support.
                \item Built-in Git version control.
            \end{itemize}
        \end{column}
    \end{columns}
    
    \vspace{1em}
    \centering
    \textit{VS Code is industry standard and free.}
\end{frame}

% -----------------------------------------------------------------------------
% About the Students
% -----------------------------------------------------------------------------
\begin{frame}{About You}
    \centering
    \Large Let's get to know you.
    
    \vspace{1em}
    \normalsize
    \begin{itemize}
        \item Name and academic/professional background?
        \item Career goals in Data Science/ML?
        \item Main learning objectives for this course?
        \item Questions/Concerns?
    \end{itemize}
\end{frame}

% -----------------------------------------------------------------------------
% Plan for Today
% -----------------------------------------------------------------------------
\begin{frame}{Plan for Today}
    \begin{block}{1. Working with Data in Jupyter}
        \begin{itemize}
            \item \textbf{Why Jupyter?} The standard for prototyping.
            \item \textbf{Preprocessing:} Why cleaning, encoding, and scaling are mandatory.
            \item \textbf{Reproducibility:} Systematic workflows in Python.
        \end{itemize}
    \end{block}
    
    \vspace{1em}
    
    \begin{exampleblock}{2. Hands-on Material}
        \begin{itemize}
            \item Lecture Notebook: \texttt{01\_Working\_with\_data\_in\_jupyter...ipynb}
            \item Exercise Notebook: \texttt{01\_Data\_preprocessing\_titanic.ipynb}
        \end{itemize}
    \end{exampleblock}
\end{frame}

% -----------------------------------------------------------------------------
% ML Workflow
% -----------------------------------------------------------------------------
\begin{frame}{The Machine Learning Workflow}
    \begin{enumerate}
        \item \textbf{Define \& Collect:} Clarify task, gather data.
        \pause
        \item \textbf{Preprocessing:} Clean, transform, encode.
        \pause
        \item \textbf{Model Selection:} Choose algorithm (Linear, Tree, etc.).
        \pause
        \item \textbf{Training \& Tuning:} Fit model, optimize hyperparameters.
        \pause
        \item \textbf{Evaluation:} Test on unseen data, deploy.
    \end{enumerate}
\end{frame}

% -----------------------------------------------------------------------------
% Traditional ML vs Deep Learning (Table Slide)
% -----------------------------------------------------------------------------
\begin{frame}{Comparison: Traditional ML vs. Deep Learning}
    \centering
    \small
    \renewcommand{\arraystretch}{1.5} % Adds vertical padding
    
    \begin{tabular}{p{2.5cm} p{5cm} p{5cm}}
        \toprule
        \textbf{Aspect} & \textbf{Traditional ML} & \textbf{Deep Learning} \\
        \midrule
        \textbf{Features} & Manual engineering (domain knowledge) & Automated extraction (representation learning) \\
        \textbf{Data Needs} & Small/Medium structured data & Large datasets; Unstructured (Images/Text) \\
        \textbf{Complexity} & Lower (Linear, Trees, SVM) & High (Deep Neural Networks) \\
        \textbf{Interpretability} & Transparent (White-box) & Opaque (Black-box) \\
        \bottomrule
    \end{tabular}
\end{frame}

% -----------------------------------------------------------------------------
% Overfitting vs Underfitting
% -----------------------------------------------------------------------------
\begin{frame}{Overfitting vs. Underfitting}
    \begin{columns}[T]
        \begin{column}{0.48\textwidth}
            \begin{block}{Underfitting (High Bias)}
                \begin{itemize}
                    \item Model is too simple.
                    \item Misses underlying trends.
                    \item \textbf{Fix:} More complex model, new features.
                \end{itemize}
            \end{block}
        \end{column}
        
        \begin{column}{0.48\textwidth}
            \begin{alertblock}{Overfitting (High Variance)}
                \begin{itemize}
                    \item Fits noise, not signal.
                    \item Fails on unseen data.
                    \item \textbf{Fix:} Regularization, more data, simpler model.
                \end{itemize}
            \end{alertblock}
        \end{column}
    \end{columns}
\end{frame}

% -----------------------------------------------------------------------------
% Data Splits
% -----------------------------------------------------------------------------
\begin{frame}{Train, Validation, and Test Splits}
    \begin{itemize}
        \item \textbf{Training Set (60-80\%):} Fit model parameters.
        \item \textbf{Validation Set (10-20\%):} Hyperparameter tuning \& model selection.
        \item \textbf{Test Set (10-20\%):} Final evaluation on "unseen" data.
    \end{itemize}
    
    \vspace{1em}
    \centering
    \rule{0.8\textwidth}{0.5pt} % Horizontal line
    
    \vspace{0.5em}
    \textbf{Crucial:} Use random splits or stratification to ensure representative samples.
\end{frame}

% -----------------------------------------------------------------------------
% Why Learn This?
% -----------------------------------------------------------------------------
\begin{frame}{Why Genuine Learning Matters}
    \textit{You have unlimited resources (AI, Home Exam). Why learn the math?}
    
    \vspace{1em}
    
    \begin{itemize}
        \item \textbf{Skill Building:} Solve unseen problems tomorrow, not just today's.
        \item \textbf{Critical Thinking:} Detect AI hallucinations.
        \item \textbf{Career:} Employers need people who can \textit{fix} AI, not just use it.
        \item \textbf{Exam Rigor:} You must \textbf{justify} every step. "The AI said so" is not a valid answer.
    \end{itemize}
\end{frame}

% -----------------------------------------------------------------------------
% Final Slide
% -----------------------------------------------------------------------------
\begin{frame}{Final Expectations}
    \centering
    \Large
    \begin{itemize}
        \item Engage actively.
        \item Be ready to talk through your solutions.
        \item \textbf{Your genuine knowledge is what matters.}
    \end{itemize}
\end{frame}

\end{document}
