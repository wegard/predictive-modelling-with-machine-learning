\documentclass[12pt, a4paper]{article}
%\usepackage{geometry}
\usepackage[inner=1.0cm,outer=1.0cm,top=2.5cm,bottom=2.5cm]{geometry}
\pagestyle{empty}
\usepackage{graphicx}
\usepackage{fancyhdr, lastpage, bbding, pmboxdraw, tabularx}
\usepackage[usenames,dvipsnames]{color}
\definecolor{darkblue}{rgb}{0,0,.6}
\definecolor{darkred}{rgb}{.7,0,0}
\definecolor{darkgreen}{rgb}{0,.6,0}
\definecolor{red}{rgb}{.98,0,0}
\usepackage[colorlinks,pagebackref,pdfusetitle,urlcolor=darkblue,citecolor=darkblue,linkcolor=darkred,bookmarksnumbered,plainpages=false]{hyperref}
\renewcommand{\thefootnote}{\fnsymbol{footnote}}
\newcommand{\RNum}[1]{\uppercase\expandafter{\romannumeral #1\relax}}
\pagestyle{fancyplain}
\fancyhf{}
\lhead{ \fancyplain{}{\textsc{GRA 4160} }}
\chead{ \fancyplain{}{} }
\rhead{ \fancyplain{}{\textsc{Course description} }}
%\rfoot{\fancyplain{}{page \thepage\ of \pageref{LastPage}}}
\fancyfoot[RO, LE] {page \thepage\ of \pageref{LastPage} }
\thispagestyle{plain}

%%%%%%%%%%%% LISTING %%%
\usepackage{listings}
\usepackage{caption}
\DeclareCaptionFont{white}{\color{white}}
\DeclareCaptionFormat{listing}{\colorbox{gray}{\parbox{\textwidth}{#1#2#3}}}
\captionsetup[lstlisting]{format=listing,labelfont=white,textfont=white}
\usepackage{verbatim} % used to display code
\usepackage{fancyvrb}
\usepackage{acronym}
\usepackage{amsthm}
\VerbatimFootnotes % Required, otherwise verbatim does not work in footnotes!
\definecolor{OliveGreen}{cmyk}{0.64,0,0.95,0.40}
\definecolor{CadetBlue}{cmyk}{0.62,0.57,0.23,0}
\definecolor{lightlightgray}{gray}{0.93}
\lstset{
%language=bash,                          % Code langugage
basicstyle=\ttfamily,                   % Code font, Examples: \footnotesize, \ttfamily
keywordstyle=\color{OliveGreen},        % Keywords font ('*' = uppercase)
commentstyle=\color{gray},              % Comments font
numbers=left,                           % Line nums position
numberstyle=\tiny,                      % Line-numbers fonts
stepnumber=1,                           % Step between two line-numbers
numbersep=5pt,                          % How far are line-numbers from code
backgroundcolor=\color{lightlightgray}, % Choose background color
frame=none,                             % A frame around the code
tabsize=2,                              % Default tab size
captionpos=t,                           % Caption-position = bottom
breaklines=true,                        % Automatic line breaking?
breakatwhitespace=false,                % Automatic breaks only at whitespace?
showspaces=false,                       % Dont make spaces visible
showtabs=false,                         % Dont make tabls visible
columns=flexible,                       % Column format
morekeywords={__global__, __device__},  % CUDA specific keywords
}

%%%%%%%%%%%%%%%%%%%%%%%%%%%%%%%%%%%%
\begin{document}
\begin{center}
  {\LARGE \textsc{Predictive modelling with machine learning}}
\end{center}
\begin{center}
  Semester: Spring 2025\\
  %Program: Bachelor of Digital Business\\
  Course code: GRA 4160\\
  BI Norwegian Business School\\
  {\small This version: \today}
\end{center}
%\date{September 26, 2014}

\begin{center}
  \rule{\textwidth}{0.4pt}
  \begin{minipage}[t]{\textwidth}
    \medskip
    \begin{tabularx}{\linewidth}{XXXX}
      \textbf{Instructor:}   & Vegard H. Larsen                                       & \textbf{Contact:}      & \href{mailto:vegard.h.larsen@bi.no}{vegard.h.larsen@bi.no} \medskip \\
      %\textbf{Office:}       & B3Y-075                                                & \textbf{Office Hours:} & Wednesdays 13-15                                                    \\
      \textbf{Course Pages:} & \href{https://bi.itslearning.com/}{bi.itslearning.com} &                        & \medskip                                                            \\
    \end{tabularx}\medskip
  \end{minipage}
  \rule{\textwidth}{0.4pt}
\end{center}
\vspace{.2cm}
\setlength{\unitlength}{1in}
\renewcommand{\arraystretch}{2}

\noindent
%{\bf This course description is preliminary and subject to change.}

\vskip.25in
\noindent
In this course we focus on basic machine learning methods, both for supervised and unsupervised learning.
The course covers both the theory and the practice of machine learning using Python.

%\vskip.25in
%\noindent
%\textbf{\large Course Objectives} \\

\vskip.25in
\noindent\textbf{\large Compulsory Readings}
\begin{enumerate}
  \item The elements of statistical learning: data mining, inference, and prediction by \it{Trevor J. Hastie; Robert J. Tibshirani; Jerome Friedman}
        (\href{https://hastie.su.domains/Papers/ESLII.pdf}{Link})
\end{enumerate}

\vskip.25in
\noindent\textbf{\large Mini project} \\
You will work on a mini project in groups of 3 students.
The mini project will be a small-scale project where you will apply the methods you learn in the course to a real-world problem.

\noindent
To complete the project, you will need to perform the following steps:
\begin{itemize}
  \item Identify a problem or research question that you would like to address.
  \item Collect and preprocess data for your model.
  \item Choose a machine learning algorithm (or several) and implement it in your project.
  \item All the groups can book two meetings with me to discuss the project and get some feedback either in person or on Zoom during the semester.
  \item Make a presentation summarizing your findings and discussing the results.
  \item Present for the class.
\end{itemize}

\noindent
The project is an important part of the course, as it allows you to demonstrate your understanding of machine learning concepts and techniques,
and to apply them in a practical setting.
It is also an opportunity for you to explore a topic of interest and to develop your skills as a data scientist.
The project will be graded and will count for 30\% of your final grade.

\newpage
\noindent\textbf{\large Lecture Plan}
-- Lectures are Thursdays from 12:00--14:45 in C2-055. \\
%-- The lecture plan is preliminary and subject to change. \\
\begin{enumerate}
  \item[] \underline{Lecture 1:} [January 9th] \underline{\bf Introduction and data preprocessing}
    {\small
      \begin{itemize}
        \item Ch. 1 in ESL
        \item Notebooks: \texttt{01\_Working\_with\_data\_in\_jupyter\_notebooks.ipynb}
        \item Exercise: \texttt{01\_Data\_preprocessing\_titanic.ipynb}
      \end{itemize}
    }
  \item[] \underline{Lecture 2:} [January 16th] \underline{\bf Machine learning basics and supervised learning}
    {\small
      \begin{itemize}
        \item Ch. 2 and 6.6 in ESL
        \item Notebooks: \texttt{03\_OLS.ipynb}, \texttt{04\_Supervised\_learning\_with\_kNN.ipynb}
        \item Exercise: \texttt{02\_Spam\_filtering\_with\_naive\_bayes.ipynb}
      \end{itemize}
    }
  \item[] \underline{Lecture 3:} [January 23th] \underline{\bf LDA and regularised regression analysis}
    {\small
      \begin{itemize}
        \item Ch. 3 and 5 in ESL
        \item Notebooks: \texttt{05\_Linear\_discriminant\_analysis.ipynb}, \texttt{06\_Regularised\_regressions.ipynb}
        \item Exercise: \texttt{03\_Predicting\_house\_prices.ipynb}
      \end{itemize}
    }
\item [] \underline{Lecture 4:} [January 30th] \underline{\bf Classification analysis}
        {\small
          \begin{itemize}
            \item Ch. 4 and 9 in ESL
            \item Notebooks \texttt{07\_Logistic\_regression.ipynb}, \texttt{08\_Decision\_trees.ipynb}
            \item Exercise: \texttt{04\_Recognizing\_handwritten\_digits.ipynb}
          \end{itemize}
        }
  \item[] \underline{Lecture 5:} [February 6th] \underline{\bf Model selection, evaluation and assessment}
    {\small
      \begin{itemize}
        \item Ch. 7 in ESL
        \item Notebooks: \texttt{09\_IC\_and\_CV.ipynb}, \texttt{10\_Bias\_variance\_tradeoff.ipynb}
        \item Exercise: \texttt{05\_Model\_selection\_evaluation\_and\_assessment.ipynb}
      \end{itemize}
    }
  \item[] \underline{Lecture 6:} [February 13th] \underline{\bf Unsupervised learning}
    {\small
      \begin{itemize}
        \item Ch. 13 and 14 in ESL
        \item Notebooks: \texttt{11\_PCA.ipynb}, \texttt{12\_K-means.ipynb}
        \item Exercise: \texttt{06\_Text\_classification.ipynb}
      \end{itemize}
    }
  \item[] \underline{Lecture 7:} [February 20th] \underline{\bf Ensemble methods}
    {\small
      \begin{itemize}
        \item Ch. 8, 10 and  16 in ESL
        \item Notebooks \texttt{13\_Introducing\_ensemble\_methods.ipynb}, \texttt{14\_Bagging.ipynb}
        \item Exercise: \texttt{07\_Predicting\_income.ipynb}
      \end{itemize}
    }
  \item[] \underline{Lecture 8:} [February 27th] \underline{\bf Random forests}
    {\small
      \begin{itemize}
        \item Ch. 15 in ESL
        \item Notebooks: \texttt{15\_Random\_forests.ipynb}, \texttt{16\_ExtraTrees.ipynb}
        \item Exercise: \texttt{08\_Predicting\_credit\_risk.ipynb}
      \end{itemize}
    }
  \item[] \underline{Lecture 9:} [March 6th] \underline{\bf Introduction to Deep Learning}
    {\small
      \begin{itemize}
        \item Chapter 11 in ESL
        \item Notebooks: \texttt{17a\_Gradient\_Decent.ipynb}, \texttt{17b\_Backpropagation.ipynb}, \texttt{18\_NN\_Basics.ipynb}
        \item Exercise: \texttt{09\_Neural\_Networks\_with\_PyTorch.ipynb}
      \end{itemize}
    }
  \item[] \underline{Lecture 10:} [March 13th] \underline{\bf Traditional ML vs Deep Learning}
    {\small
      \begin{itemize}
        \item Ch. 11 in ESL
        \item Notebooks: TBA
        \item Exercise: TBA
      \end{itemize}
    }
  \item[] \underline{Lecture 11:} [March 20th] {\bf Presentation and defense of the mini projects}
  \item[] \underline{Lecture 12:} [March 27th] {\bf Summing up and Q\&A}
\end{enumerate}

\vskip.25in
\noindent\textbf{\large Final exam}
\begin{itemize}
  \item Exam format: 30 hour take-home exam
  \item Start: 29.05.2023 at 09:00
  \item Submission deadline: 30.05.2023 at 15:00
  \item Weight: 70\%
\end{itemize}

\vskip.25in
\noindent\textbf{\large Software} \\
In this course we will use Python.
I recommend installing Python through the \href{https://www.anaconda.com/}{Anaconda} distribution.
You can run into some problems if you don't have the correct version of the software installed.
Try to install the latest version you find.
Python 2 will not be supported.

\vskip.25in
\noindent
Most of the material will be in the form of Jupyter notebooks.
I also recommend installing an integrated development environment (IDE) for Python.
Two good options are \href{https://www.jetbrains.com/pycharm/}{PyCharm} and \href{https://code.visualstudio.com}{VS code}.
Both IDEs are free for students, and they support Jupyter notebooks.

\vskip.25in
\noindent
We will also rely on some third party Python libraries. I recommend spending some time familiarizing yourself with the following libraries:

\begin{enumerate}
  \item Scikit-learn: \href{https://scikit-learn.org/stable/}{https://scikit-learn.org/stable/}
  \item Pandas: \href{https://pandas.pydata.org/}{https://pandas.pydata.org/}
  \item Matplotlib: \href{https://matplotlib.org/}{https://matplotlib.org/}
  \item PyTorch: \href{https://pytorch.org/}{https://pytorch.org/}
        %\item Tensorflow: \href{https://www.tensorflow.org/}{https://www.tensorflow.org/}
\end{enumerate}

%\vskip.15in
%\noindent\textbf{\large Supplementary Material} \\%\footnotemark
%Here are some interesting resources that will be useful during the course.
%This is not mandatory readings.
%\begin{itemize}
%\item \href{https://docs.python.org/3/tutorial/}{Python tutorial}
%\item \href{https://jupyter-notebook.readthedocs.io/}{Jupyter Notebook documentation}
%\item \href{https://numpy.org/doc/}{NumPy documentation}
%\item \href{https://pandas.pydata.org/docs/}{Pandas documentation}
%\item \href{https://matplotlib.org/stable/users/index}{Matplotlib documentation}
%\item \href{https://docs.python.org/3/library/sqlite3.html}{SQLite Python API documentation}
%\end{itemize}

\end{document}