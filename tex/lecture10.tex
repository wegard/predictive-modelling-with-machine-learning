% !TEX TS-program = XeLaTeX
% !TEX spellcheck = en-US
\documentclass[aspectratio=169]{beamer}

\usetheme{example}

\title{Lecture 10:\\ Traditional ML vs Deep Learning}
\institute{GRA4160: Predictive modelling with machine learning}
\date{March 13th, 2025}
\author{Vegard Høghaug Larsen}

\begin{document}

\maketitle

\frame{
\frametitle{Plan for today:}
\begin{itemize}
	\item Traditional ML vs Deep Learning - Overview
	\item Practical applications and model comparisons
	\item Tools used in practice
	\item Emerging trends in machine learning
	\item Recap of exercise from last week and some more PyTorch
	%\item Visualization resources
\end{itemize}}

\frame{
\frametitle{Traditional ML vs Deep Learning}
\textbf{Traditional ML Models:}
\begin{itemize}
	\item Algorithms: $k$-NN, decision trees, logistic regression, SVMs
	\item Manual feature selection required
	\item Interpretable, good for structured data, smaller datasets
	\item Example: Decision trees as a flowchart
\end{itemize}
\textbf{Deep Learning:}
\begin{itemize}
	\item Multi-layer neural networks
	\item Automatic feature learning from data
	\item Excellent with large datasets, images, text, audio
	\item High accuracy, high computational cost, low interpretability
	\item Often "black-box" models
\end{itemize}
}

\frame{
\frametitle{When to Choose Which?}
\begin{itemize}
	\item \textbf{Traditional ML}:
	\begin{itemize}
		\item Limited data
		\item Interpretability critical (e.g., credit decisions)
	\end{itemize}
	\item \textbf{Deep Learning}:
	\begin{itemize}
		\item Large datasets available
		\item Complex, perceptual tasks (vision, NLP)
		\item Computational resources available
	\end{itemize}
	\item Practice: Start simple (Logistic, Random Forest) then move to DL if needed
\end{itemize}}

\frame{
\frametitle{k-NN vs Neural Networks (Recommenders)}
\textbf{k-NN:}
\begin{itemize}
	\item Simple, memory-based, no explicit training
	\item Good for small-scale recommendations
	\item Slow at scale, limited to direct similarity
\end{itemize}
\textbf{Neural Networks:}
\begin{itemize}
	\item Learns embeddings for user-item interactions
	\item Scalable, handles diverse data, highly accurate
	\item Used by Netflix, YouTube
\end{itemize}}

\frame{
\frametitle{Logistic Regression vs DL (Text Classification)}
\textbf{Logistic Regression:}
\begin{itemize}
	\item Baseline, simple linear model, interpretable
	\item Effective with engineered features (e.g. TF-IDF)
	\item Limited nuance in complex texts
\end{itemize}
\textbf{Deep Learning (CNNs, LSTMs, Transformers):}
\begin{itemize}
	\item Surpassed traditional methods on NLP tasks
	\item Learns word/context embeddings
	\item More data-intensive, computationally demanding
\end{itemize}}

\frame{
\frametitle{Decision Trees vs Neural Networks (Medical)}
\textbf{Decision Trees/Ensembles:}
\begin{itemize}
	\item Highly interpretable, suitable for small datasets
	\item Used for simple diagnostic rules
	\item Easy to validate by clinicians
\end{itemize}
\textbf{Deep Learning:}
\begin{itemize}
	\item Human-level accuracy (X-rays, pathology slides)
	\item Harder to interpret, "black-box"
	\item Explainability (XAI) essential in practice
\end{itemize}}

\frame{
\frametitle{Random Forests vs CNNs (Images)}
\textbf{Random Forests:}
\begin{itemize}
	\item Requires extensive feature engineering
	\item Limited scalability and accuracy
\end{itemize}
\textbf{CNNs:}
\begin{itemize}
	\item Revolutionized image recognition
	\item Learn hierarchical features directly from pixels
	\item State-of-the-art accuracy, computationally demanding
\end{itemize}}

\frame{
\frametitle{Ensemble Methods vs DL (Finance)}
\textbf{Ensemble Methods:}
\begin{itemize}
	\item Good with structured/tabular financial data
	\item Robust with smaller datasets
	\item Interpretable, less tuning required
\end{itemize}
\textbf{Deep Learning:}
\begin{itemize}
	\item Powerful with large historical datasets
	\item Captures subtle, complex interactions
	\item Harder to interpret, extensive tuning required
\end{itemize}}

\frame{
\frametitle{Tools in Practice}
\textbf{Scikit-learn:}
\begin{itemize}
	\item Traditional ML, quick prototyping, easy to use
	\item CPU-based, integrates with pandas, numpy
\end{itemize}
\textbf{TensorFlow/PyTorch:}
\begin{itemize}
	\item Deep learning frameworks, GPU acceleration
	\item TensorFlow: Production deployments
	\item PyTorch: Research, dynamic graphs, HuggingFace integration
\end{itemize}}

\frame{
\frametitle{Emerging Trends in ML}
\begin{itemize}
	\item \textbf{Hybrid Models}: Combining DL features with interpretable ML
	\item \textbf{Transfer/Self-supervised Learning}: Leveraging large, pre-trained models
	\item \textbf{Foundation Models}: GPT, Stable Diffusion, general-purpose models
	\item \textbf{AutoML/No-Code ML}: Automated model building, increased accessibility
\end{itemize}}

% \frame{
% \frametitle{Visualization Resources}
% \textbf{Interactive:}
% \begin{itemize}
% 	\item TensorFlow Playground: \url{https://playground.tensorflow.org}
% 	\item Decision trees, SHAP value visualizations
% \end{itemize}
% \textbf{Static:}
% \begin{itemize}
% 	\item Feature Engineering vs Feature Learning (Arivis blog)
% 	\item "Deep Learning" (Goodfellow et al.) visualizations
% 	\item Distill.pub interactive articles
% \end{itemize}}

\end{document}
