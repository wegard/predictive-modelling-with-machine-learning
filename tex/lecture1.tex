% !TEX TS-program = XeLaTeX
% !TEX spellcheck = en-US
\documentclass[aspectratio=169]{beamer}

\usetheme{example}

\title{Lecture 1: \\ Introduction and data preprocessing}
\institute{GRA 4160: Predictive Modelling with Machine Learning}
\date{January 9th 2025}
\author{Vegard H\o ghaug Larsen}

\begin{document}

\maketitle

\frame{
	\frametitle{About me:}
	Vegard H\o ghaug Larsen (Department of Data Science and Analytics)\\
	Background: PhD in Economics followed by six years in Norges Bank
	$\;$\\
	Email: vegard.h.larsen@bi.no\\
	%$\;$\\
	%Office hours: Wednesday 13:00-15:00 (or by appointment)\\
	$\;$\\
	Office: B3Y-075
}

\frame{
	\frametitle{Syllabus}

The text book for this course is:\\
	\begin{enumerate}
		\item[] \textit{The elements of statistical learning: data mining, inference, and prediction} by \it{Trevor J. Hastie; Robert J. Tibshirani; Jerome Friedman}
		(\href{https://hastie.su.domains/Papers/ESLII.pdf}{Link})
	\end{enumerate}

For a more gentle introduction to many of the same subjects, I recommend the following book:\\
	\begin{enumerate}
		\item[] \textit{An introduction to statistical learning} by \it{Gareth James; Daniela Witten; Trevor Hastie; Robert Tibshirani}
		(\href{https://www.statlearning.com/}{Link})
	\end{enumerate}
}

\frame{
	\frametitle{Prerequisites/What I expect that you know}

	\begin{itemize}
		\item Linear algebra and multivariate calculus
		\item Probability and statistics
		\item Programming in Python (NumPy, Pandas, matplotlib)
		\item Object oriented programming
	\end{itemize}
}

\frame{
\frametitle{Mini Project}
\begin{itemize}
	\item You will work on a mini project in groups of 2-4 students.
	\item The mini project will be a small-scale project where you will apply the methods you've learned in the course to a real-world problem.
	\item All groups can schedule two meetings with me to discuss the project and receive feedback, either in person or via Zoom, during the semester.
	\item Present the project to the rest of the class.
	\item Optional but recommended: Create a GitHub repository for the project where you can upload the code and a short report/documentation of the project.
	\item The project will be graded (letter grade) and will account for 30\% of the final grade.
	\item You may also be asked some questions related to the project on the final exam.
\end{itemize}
}

\frame{
\frametitle{Course Structure}
\begin{itemize}
\item I will strive to make the lectures as interactive as possible, with traditional lectures kept to a minimum.
\item Often, I will begin with a few slides to introduce a topic, but the bulk of the lectures will be dedicated to working through examples and exercises.
\item I will typically provide two notebooks with course materials and one notebook containing exercises that we will tackle during the lectures.
\item These notebooks will be available on itslearning before the class, and I expect you to have reviewed them beforehand.
\item Following or during the lecture, I will provide solutions to the exercises.
\end{itemize}
}

\frame{
	\frametitle{Using Generative AI as a Learning Tool}

	\begin{itemize}
		\item Generative AI, a powerful technology, is available for your use during this course.
		\item Our goal is to use this tool as a means of learning and exploration.
		\item It should be used to enhance your understanding, creativity, and problem-solving skills.
		\item Emphasize using generative AI for experimentation, not just content generation.
		\item Explore its capabilities to gain insights, generate ideas, and aid in your learning journey.
		\item Think of it as a tool to augment your skills, not replace them.
		\item I encourage responsible and ethical use of generative AI throughout the course.
		\item You should have access to GPT-UiO, that satisfy GDPR requirements.
		\item I also recommend GitHub Copilot, but it is not available for free.
	\end{itemize}
}

\frame{
	\frametitle{You?}
	\begin{itemize}
		\item What is your name background?
		\pause
		\item Do you have any specific career goals in mind?
		\pause
		\item What do you hope to learn from this course?
		\pause
		\item Do you have any specific questions about the course structure or content?
	\end{itemize}
}

	\frame{
		\frametitle{Plan for today}

	\underline{\bf Introduction and data preprocessing}

		\begin{itemize}
			\item Data preprocessing is a very important part of any data science project.
			\item We will have to use some type of preprocessing in almost every project we do.
			\item In this lecture we will go through some of the most common preprocessing techniques and how to implement them in Python.
		\end{itemize}

		\bigskip
		\pause

	Material:\\
         \begin{itemize}
             \item Notebooks: \texttt{01\_Intro\_to\_GRA4160.ipynb}, \texttt{02\_Data\_preprocessing.ipynb}
             \item Exercise: \texttt{01\_Data\_preprocessing\_titanic.ipynb}
         \end{itemize}
	}

\frame{
	\frametitle{Scikit-learn}

	\url{https://scikit-learn.org/stable/}\\
}

\frame{
	\frametitle{Keras}

	\url{https://keras.io/}\\
}

\frame{
\frametitle{Conda Virtual Environments}

	\begin{itemize}
		\item Conda is a package manager that allows the creation of virtual environments.
		\item A virtual environment is a self-contained directory tree that includes a Python installation for a specific Python version, along with various additional packages.
		\item Virtual environments are employed to isolate different projects from one another.
		\item This is beneficial because distinct projects may necessitate different versions of the same package.
	\end{itemize}
	
	\pause
	My Setup: Environment Named \texttt{GRA4160}\\
	\begin{itemize}
		\item Python 3.10
		\item Jupyter
		\item Scikit-learn
		\item Pandas
		\item Matplotlib and Seaborn
	\end{itemize}
	
	I have not installed Keras and Tensorflow in my environment.
	I will create a new environment for neural networks later.
}

\end{document}