% !TEX TS-program = XeLaTeX
% !TEX spellcheck = en-US
\documentclass[aspectratio=169]{beamer}

\usetheme{example}

\title{Lecture 1: \\ Introduction and working with data}
\institute{GRA 4160: Predictive Modelling with Machine Learning}
\date{January 16th 2025}
\author{Vegard H\o ghaug Larsen}

\begin{document}

\maketitle

\frame{
	\frametitle{About me:}
	Vegard H\o ghaug Larsen (Department of Data Science and Analytics)\\
	Background: PhD in Economics followed by six years in Norges Bank
	$\;$\\
	Email: vegard.h.larsen@bi.no\\
	%$\;$\\
	%Office hours: Wednesday 13:00-15:00 (or by appointment)\\
	$\;$\\
	Office: B3Y-075
}

\frame{
	\frametitle{Syllabus}

The text book for this course is:\\
	\begin{enumerate}
		\item[] \textit{The elements of statistical learning: data mining, inference, and prediction} by \it{Trevor J. Hastie; Robert J. Tibshirani; Jerome Friedman}
		(\href{https://hastie.su.domains/Papers/ESLII.pdf}{Link})
	\end{enumerate}

\bigskip
For a more gentle introduction to many of the same subjects, I recommend the following book:\\
	\begin{enumerate}
		\item[] \textit{An introduction to statistical learning} by \it{Gareth James; Daniela Witten; Trevor Hastie; Robert Tibshirani}
		(\href{https://www.statlearning.com/}{Link})
	\end{enumerate}
}

\frame{
	\frametitle{Prerequisites / What I Expect You to Know}

	\begin{itemize}
		\item \textbf{Mathematics:} Basic linear algebra and multivariate calculus
		\item \textbf{Statistics:} Fundamental probability and statistical concepts
		\item \textbf{Python Programming:} Experience with NumPy, Pandas, and Matplotlib
		\item \textbf{Object-Oriented Programming:} Ability to write and understand OOP structures in Python
	\end{itemize}
}


\frame{
	\frametitle{About this Course}

	\begin{itemize}
		\item Focused on \textbf{predictive modeling} using traditional (pre-deep learning) machine learning methods.
		\item Covers fundamental algorithms such as linear and logistic regression, decision trees, and ensemble methods.
		\item Emphasis on \textbf{hands-on practice}: you will write code from scratch to implement and understand key algorithms.
		\item Provides practical insights into common workflows: data preprocessing, model training, hyperparameter tuning, and model evaluation.
		\item Prepares you to critically evaluate and improve models for real-world predictive tasks.
	\end{itemize}
}


\frame{
\frametitle{Mini Project}

\begin{itemize}
	\item \textbf{Group Work:} Form groups of 2--4 students. I will add a excel sheet on itslearning that you should use to form groups within next week. 
	\item \textbf{Objective:} Apply the learned methods to a real-world data problem on a smaller scale.
	\item \textbf{Feedback Sessions:} Each group can schedule up to two meetings (in-person or via Zoom) for guidance and feedback.
	\item \textbf{Presentation:} Present your work to the class at the end of the semester.
	\item \textbf{Documentation:} (Optional but recommended) Use a GitHub repository to share code and a short project report.
	\item \textbf{Evaluation:} The project is graded (letter grade) and accounts for 30\% of the final grade.
	\item \textbf{Exam Integration:} Be prepared to answer related questions during the final exam.
\end{itemize}
}


\frame{
\frametitle{Course Structure}

\begin{itemize}
	\item \textbf{Interactive Format:} Lectures will be as interactive as possible, minimizing traditional lecture-style teaching.
	\item \textbf{Topic Introductions:} I will often start with a few slides to introduce a concept, then focus on live coding, examples, and exercises.
	\item \textbf{Course Materials:} Expect one to two notebooks with background materials and one exercise notebook per session.
	\item \textbf{Pre-class Preparation:} All notebooks will be available on Itslearning before each class, and you are expected to review them in advance.
	\item \textbf{Project Guidance:} We will also devote time in some lectures to discuss the mini project and offer guidance on your approach.
\end{itemize}
}


\frame{
	\frametitle{Using Generative AI as a Learning Tool}

	\begin{itemize}
		\item Generative AI is a powerful resource for exploration, idea generation, and problem-solving.
		\item Treat it as a learning partner rather than a replacement for your own reasoning skills.
		\item Use it to enhance your understanding, spark creativity, and experiment with new approaches.
		\item Focus on leveraging it as a supportive tool: ask questions, seek clarifications, and discover alternative solutions.
		\item You can access \textbf{GPT-UiO}, a GDPR-compliant platform provided by UiO, for free.
		\item \textbf{GitHub Copilot} is also recommended. You should be able to get an educational license for free.
		\item Always apply critical thinking, ethical considerations, and proper attribution when using these tools.
	\end{itemize}
}


\frame{
	\frametitle{About You}
	\begin{itemize}
		\item Please introduce yourself: What is your name and your academic/professional background?
		\pause
		\item Do you have any specific career goals related to data science, analytics, or machine learning?
		\pause
		\item What are your main learning objectives or expectations for this course?
		\pause
		\item Are there any questions or concerns you have about the course structure, topics, or logistics?
	\end{itemize}
}


\frame{
	\frametitle{Plan for Today}
	
	\underline{\bf Working with Data in Jupyter Notebooks:}
	
	\begin{itemize}
		\item Understand why \textbf{Jupyter notebooks} have become a standard tool for data exploration and ML prototyping.
		\item Recognize that \textbf{data preprocessing} (cleaning, transforming, encoding, and scaling) is essential for high-quality predictive modeling.
		\item Learn how to systematically apply preprocessing techniques in Python, ensuring reproducibility and consistency across projects.
		\item Gain hands-on experience implementing common techniques (handling missing values, encoding categories, scaling features) to prepare data for modeling.
	\end{itemize}
	
	\bigskip
	\pause

	\underline{\bf Material:}
	\begin{itemize}
		\item Lecture Notebook: \texttt{01\_Working\_with\_data\_in\_jupyter\_notebooks.ipynb}
		\item Exercise Notebook: \texttt{01\_Data\_preprocessing\_titanic.ipynb}
	\end{itemize}
}


\end{document}