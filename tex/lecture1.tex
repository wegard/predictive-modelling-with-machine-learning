% !TEX TS-program = XeLaTeX
% !TEX spellcheck = en-US
\documentclass[aspectratio=169]{beamer}

\usetheme{bi}

\title{Lecture 1: \\ Introduction and data preprocessing}
\institute{GRA4160: Advanced Regression and Classification Analysis,\\
Ensemble Methods and Neural Networks}
\date{January 11th 2024}
\author{Vegard H\o ghaug Larsen}

\begin{document}

\maketitle

\frame{
	\frametitle{About me:}
	Vegard H\o ghaug Larsen (Department of Data Science and Analytics, BI)\\
	$\;$\\
	Email: vegard.h.larsen@bi.no\\
	$\;$\\
	Office hours: Wednesday 12:00-13:00 (or by appointment)\\
	$\;$\\
	Office: B3Y-097
}

\frame{
	\frametitle{Syllabus}

The text book for this course is:\\
	\begin{enumerate}
		\item[] \textit{The elements of statistical learning: data mining, inference, and prediction} by \it{Trevor J. Hastie; Robert J. Tibshirani; Jerome Friedman}
		(\href{https://hastie.su.domains/Papers/ESLII.pdf}{Link})
	\end{enumerate}

For a more gentle introduction to many of the same subjects, I recommend the following book:\\
	\begin{enumerate}
		\item[] \textit{An introduction to statistical learning} by \it{Gareth James; Daniela Witten; Trevor Hastie; Robert Tibshirani}
		(\href{https://www.statlearning.com/}{Link})
	\end{enumerate}

}

\frame{
	\frametitle{Prerequisites/What I expect that you know}

	\begin{itemize}
		\item Linear algebra and multivariate calculus
		\item Probability and statistics
		\item Programming in Python (NumPy, pandas, matplotlib)
		\item Object oriented programming
	\end{itemize}
}

\frame{
	\frametitle{Mini project}

\begin{itemize}
	\item You will work on a mini project in groups of 3 students.
	\item The mini project will be a small-scale project where you will apply the methods you learn in the course to a real-world problem.
	\item All the groups can book two meetings with me to discuss the project and get some feedback either in person or on Zoom during the semester.
	\item Present the project for the rest of the class.
	\item The project will be graded and will count for 30\% of the final grade.
\end{itemize}
}

\frame{
	\frametitle{Course structure}
	\begin{itemize}
		\item Traditional lectures will only be used to a small extent in GRA4160.
		\item Often I will show a few slides to introduce a topic, but the main part of the lectures will be spent on working through examples and exercises.
		\item I will usually provide two notebooks with material and one notebook with an exercises that we will work on during the lectures.
		\item These notebooks will be available on itslearning before the class and I expect that you have read through them before we meet.
		\item After (or during) the lecture I will provide a solutions to the exercises.
	\end{itemize}
}


\frame{
	\frametitle{You?}
	\begin{itemize}
		\item Do you have any specific career goals in mind?
		\pause
		\item What do you hope to learn from this course?
		\pause
		\item Do you have any specific questions about the course structure?
	\end{itemize}
}

	\frame{
		\frametitle{Plan for today}

	\underline{\bf Introduction and data preprocessing}

		\begin{itemize}
			\item Data preprocessing is a very important part of any data science project.
			\item We will have to use some type of preprocessing in almost every project we do.
			\item In this lecture we will go through some of the most common preprocessing techniques and how to implement them in Python.
		\end{itemize}

		\bigskip
		\pause

	Material:\\
         \begin{itemize}
             \item Notebooks: \texttt{01\_Intro\_to\_GRA4160.ipynb}, \texttt{02\_Data\_preprocessing.ipynb}
             \item Exercise: \texttt{01\_Data\_preprocessing\_titanic.ipynb}
         \end{itemize}
	}

\frame{
	\frametitle{Scikit-learn}

	\url{https://scikit-learn.org/stable/}\\

}

\frame{
	\frametitle{Keras}

	\url{https://keras.io/}\\
}

\frame{
	\frametitle{Conda virtual environments}

	\begin{itemize}
			\item Conda is a package manager that can be used to create virtual environments.
			\item A virtual environment is a self-contained directory tree that contains a Python installation for a particular version of Python, plus a number of additional packages.
			\item Virtual environments are used to isolate different projects from each other.
			\item This is useful because different projects may require different versions of the same package.
	\end{itemize}

	\pause
	My setup: Environment called \texttt{GRA4160}\\
	\begin{itemize}
		\item Python 3.10
		\item Jupyter
		\item Scikit-learn
		\item Pandas
		\item Matplotlib and Seaborn
	\end{itemize}

	I have not installed Keras and Tensorflow in my environment.
	I will create a new environment for neural networks later.


}




\end{document}